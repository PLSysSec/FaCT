\documentclass[11pt]{article}
\usepackage{amsmath,amssymb}
\usepackage{supertabular}
\usepackage{geometry}
\usepackage{ifthen}
\usepackage{alltt}%hack
\geometry{a4paper,dvips,twoside,left=22.5mm,right=22.5mm,top=20mm,bottom=30mm}
\usepackage{color}
\usepackage{stmaryrd}
\usepackage{graphicx}
\usepackage{mathpartir}
\usepackage{mathtools}
\usepackage{etextools}

\makeatletter
    \newcommand{\DeclarePairedDelimiterCase}[2]{%
        \newcommand#1[1][]{%
            \ifthenelse{\equal{##1}{normal}}%
            {#2}%
            {%
                \ifthenelse{\equal{##1}{big}\OR\equal{##1}{Big}\OR\equal{##1}{bigg}\OR\equal{##1}{Bigg}}%
                {\expandnext{#2[}{\csname##1\endcsname}]}%
                {#2*}%        % standard case using \left and \right
            }%
        }%
    }
    \newcommand{\DeclarePairedDelimiterY}[4][Temp]{%
        \expandafter\DeclarePairedDelimiter\csname#2#1\endcsname{#3}{#4}%
        \expandnext{\expandnext{\DeclarePairedDelimiterCase}{\csname#2\endcsname}}{\csname#2#1\endcsname}%
    }
    \newcommand{\DeclarePairedDelimiterXY}[6][Temp]{%
        \expandafter\DeclarePairedDelimiterX\csname#2#1\endcsname[#3]{#4}{#5}{#6}%
        \expandnext{\expandnext{\DeclarePairedDelimiterCase}{\csname#2\endcsname}}{\csname#2#1\endcsname}%
    }
\makeatother

\DeclarePairedDelimiterY{floor}{\lfloor}{\rfloor}
\DeclarePairedDelimiterY{ceil}{\lceil}{\rceil}
\DeclarePairedDelimiterY{abs}{\lvert}{\rvert}
\DeclarePairedDelimiterY{vc}{\langle}{\rangle}

\newcommand\has\vdash
\newcommand\G\Gamma
\newcommand\F{\mathbb{F}}
\newcommand\Z{\mathbb{Z}}
\renewcommand\t\tau

\begin{document}

\subsubsection*{Type Lattice}
\begin{mathpar}

  \infer
  { n_1 < n_2 }
  { uint\vc{n_1} <: uint\vc{n_2} }

  \infer
  { n_1 < n_2 }
  { int\vc{n_1} <: int\vc{n_2} }

  \infer
  { }
  { uint\vc{n} <: int\vc{2n} }

  \infer
  { \t_1 <: \t_2 \\
    \Gamma \has e : \t_1 }
  { \Gamma \has e : \t_2 }

\end{mathpar}

\subsubsection*{Expressions}
\begin{mathpar}

  \infer[Var]
  { \G (x) = \tau }
  { \G \has x : \tau }

  \infer[Unop]
  { \G \has e : \tau \\
    \ominus : \tau \to \tau }
  { \G \has \ominus e : \tau }

  \infer[Binop]
  { \G \has e_1 : \tau_1 \\
    \G \has e_2 : \tau_2 \\
    \oplus : \tau_1 \to \tau_2 \to \tau_3 }
  { \G \has e_1 \oplus e_2 : \tau_3 }

  \infer[ArrGet]
  { \G \has e : uint\vc{max} \\
    \G \has a : array\vc{\tau,n} }
  { \G \has a[e] : \tau }

  \infer[FnCall]
  { \F (f) = fdec f(x_1:\t_1, \dots, x_n:\t_n) : \t_r \\
    \G \has e_1 : \t_1 \\ \cdots \\ \G \has e_n : \t_n }
  { \G \has f(e_1,\dots,e_n) : \t_r }

  \infer[True] { } { \Gamma \has true : bool } \\
  \infer[False] { } { \Gamma \has false : bool }

  \infer[ArrLiteral]
  { \forall i : \Gamma \has e_i : \tau }
  { \Gamma \has [ e_1, \dots, e_n ] : array\vc{\tau,n} }

  \infer[PosNumber]
  { \ell >= 0 \\
    n = \ceil{\log_2 \ell} }
  { \Gamma \has \ell : uint\vc{n} }

  \infer[NegNumber]
  { \ell < 0 \\
    n = \ceil{\log_2 \abs{\ell}}  + 1 }
  { \Gamma \has \ell : int\vc{n} }

\end{mathpar}
\end{document}
