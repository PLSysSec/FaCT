\section{Evaluation}
\label{sec:evaluation}
We were not able to evaluate our implementation due to a variety of reasons.

\cite{almeida2016}

Moreover, we were not able to perform microbenchmarks on this fucntion because
it would not have been fair. We only implemented one function that cannot stand
on it's own --- in order to truly test this, we would like to implement the
functinos needed to encrypt and decrypt, so that we can test the function is
it's full form.

TODO: Note that the branch in~\ref{fig:openssl-original} is legal since the lengths in the conditional are public.

\begin{figure*}
    \centering
    \begin{subfigure}[b]{0.49\textwidth}
		\dbox{\begin{minipage}[t]{3.1in}
\small
\begin{Verbatim}
int ssl3_cbc_remove_padding(const SSL *s, 
  SSL3_RECORD *rec, unsigned block_size, 
  unsigned mac_size) \{
 
  unsigned padding_length, good;
  const unsigned overhead = 1 + mac_size;

  if (overhead > rec->length) return 0;
  padding_length = rec->data[rec->length - 1];

  \textcolor{blue}{good = constant_time_ge(}
    \textcolor{blue}{rec->length, padding_length + overhead);}
  \textcolor{blue}{good &= constant_time_ge(}
    \textcolor{blue}{block_size, padding_length + 1);}

  \textcolor{purple}{padding_length = good & (padding_length + 1);}

  rec->length -= padding_length;
  rec->type |= padding_length << 8;

  \textcolor{GaryGreen}{return constant_time_select_int(good, 1, -1);}
\}
\end{Verbatim}
\end{minipage}
}
		\caption{The original OpenSSL function.\newline}
		\label{fig:openssl-original}
    \end{subfigure}
    \begin{subfigure}[b]{0.49\textwidth}
		\dbox{\begin{minipage}[t]{3.1in}
\small
\begin{Verbatim}
int ssl3_cbc_remove_padding(const SSL *s, 
  SSL3_RECORD *rec, unsigned block_size, 
  unsigned mac_size)\{

  unsigned padding_length, good;
  const unsigned overhead = 1 + mac_size;

  if (overhead > rec->length) return 0;
  padding_length = rec->data[rec->length - 1];

  \textcolor{purple}{good = (rec->length >= padding_length + overhead)}
    \textcolor{purple}{&& (block_size >= (padding_length + 1));}
    
  \textcolor{blue}{if(good) padding_length = padding_length + 1;}
  \textcolor{blue}{else padding_length = 0;}

  rec->length -= padding_length;
  rec->type |= padding_length << 8;

  \textcolor{red}{if(good) return 1;}
  \textcolor{red}{else return -1;}
\}
\end{Verbatim}
\end{minipage}}
		\caption{The OpenSSL function expressed more intuitively, without using their constant-time branch-avoiding routines.}
		\label{fig:openssl-intuitive}
    \end{subfigure}
    \caption{OpenSSL function used to evaluate Constanc.}\label{fig:openssl}\floatspace
\end{figure*}
